\tikzstyle{class} = [ellipse, minimum width=4.5cm, minimum height=1cm,text centered, draw=black, fill=yellow!30]
\tikzstyle{input} = [trapezium, trapezium left angle=70, trapezium right angle=110, minimum width=5cm, minimum height=2cm, text centered, draw=black, fill=blue!30]
\tikzstyle{output} = [trapezium, trapezium left angle=70, trapezium right angle=110, minimum width=5cm, minimum height=2cm, text centered, draw=black, fill=green!30]
\tikzstyle{method} = [rectangle, minimum width=5cm, minimum height=2cm, text centered, draw=black, fill=orange!30]
\tikzstyle{base} = [rectangle, rounded corners, minimum width=2cm, minimum height=2cm,text centered, draw=black, fill=red!30]
\tikzstyle{cond} = [diamond, minimum width=5cm, minimum height=2cm, text centered, draw=black, fill=white!30]
\tikzstyle{arrow} = [thick,->,>=stealth]

% Design Template
\begin{frame}
	\centering
	\frametitle{Design}

	\begin{block}{Programming Language: Python}	
		\begin{enumerate}
			\item Easy interaction with filesystem
			\item Cross-platform
			\item Light-weight
			\item Quick prototyping
		\end{enumerate}	
	\end{block}
	
	\pause
	
	\begin{block}{Input}
	\end{block}
	
	\pause
	
	\begin{block}{Output}
	\end{block}
	
\end{frame}

\begin{frame}
	\centering
	\frametitle{Design}
	\begin{block}{Program Structure}
		\begin{enumerate}
			\item 
%			\item find\_git\_report
%			\item git\_issues
%			\item write\_reports
%			\item has\_issues
%			\item get\_file
%			\item get\_file\_new
%			\item get\_unpushed
%			\item get\_commits\_behind
%			\item write\_report
		\end{enumerate}	
	\end{block}	
\end{frame}


%Design Diagram
% NEED TO UPDATE WITH NEW DESIGN DIAGRAM
\begin{frame}
	\centering
	\frametitle{Design}
	
	% Adjust dimensions of resize box as desired
	\resizebox{7cm}{7.5cm}{
	
		\begin{tikzpicture}[node distance=2cm]

% objects
\node (class) at (0,0) [class] {\texttt{Path of the Directory}};
\node (method1) at (0,-3) [method] {\texttt{find\_git\_repos(Directory)}};

\node (method2) at (0,-6) [method] {\texttt{git\_issues(Directory)}};

\node (method3) at (0,-9) [method] {\texttt{has\_issues()}};
\node (method4) at (0,-13) [method] {\texttt{write\_report()}};


% \node (method4) at (8,-13) [method] {\texttt{incPos(temp, pos, size)}};
\node (base) at (0,-16) [base] {\texttt{html report}};


% arrows
%\draw [arrow] (class) -- (method1);
\draw [arrow] (class) -- node[anchor=east] {user-input} (method1);
\draw [arrow] (method1) -- (method2);
\draw [arrow] (method2) -- (method3);
\draw [arrow] (method3) -- node[anchor=east] {return issue\_list} (method4);
\draw [arrow] (method4) -- (base);

		\end{tikzpicture}	
	} %end resizebox

\end{frame}